% !TEX root = ../../../main.tex

\toggletrue{image}
\toggletrue{imagehover}
\chapterimage{wrong_times_table}
\chapterimagetitle{\uppercase{Wrong Times Table}}
\chapterimageurl{https://xkcd.com/2313/}
\chapterimagehover{Deep in some corner of my heart, I suspect that real times tables are wrong about 6x7=42 and 8x7=56.}

\chapter{Rechnen}
\label{chapter-rechnen}

Mit einer Programmiersprache können wir natürlich auch Berechnungen durchführen. Dies war schliesslich auch die ursprüngliche Motivation für die Konstruktion von Computern\footnote{\say{to compute} kann mit rechnen oder etwas berechnen übersetzt werden. Im deutschen Sprachraum ist auch Rechner als Bezeichnung für den Computer üblich.}. In diesem Kapitel beschäftigen wir uns hauptsächlich mit den grundlegenden Rechenarten, welche wir alle aus dem Mathematikunterricht kennen. Die Lernziele lauten:

\newcommand{\rechnenLernziele}{
\begin{todolist}
\item Sie erstellen Python-Programme, in denen Sie mit Zahlen rechnen.
\item Sie erklären, was ein arithmetischer Operator ist und geben ein Beispiel.
\item Sie erklären, was ein arithmetischer Ausdruck ist und geben ein Beispiel.
\end{todolist}
}

\lernziel{\autoref{chapter-rechnen}, \nameref{chapter-rechnen}}{\protect\rechnenLernziele}

\rechnenLernziele

\section{Addition, Subtraktion, Multiplikation, Division und Potenzierung}

Wie beim Taschenrechner gibt es für jede Rechenart ein Zeichen. 

\begin{multicols}{2}
\begin{itemize}
	\item Addition: \lstinline{+}
	\item Subtraktion: \lstinline{-}
	\item Multiplikation: \lstinline{*}
	\item Division: \lstinline{/}
	\item ganzzahlige Division: \lstinline{//}
	\item Potenzierung: \lstinline{**}
\end{itemize}
\end{multicols}

\begin{definition}[Arithmetischer Operator]
Der Fachbegriff für \say{Rechenzeichen} lautet in der Programmierung arithmetischer Operator. Mit arithmetischen Operatoren notieren wir mathematische Rechnungen.
\end{definition}

Die Arithmetik ist ein Teilgebiet der Mathematik und beschreibt das Rechnen mit Zahlen.

\begin{example}
\autoref{lst-calc-print-1} zeigt Beispiele in Kombination mit \lstinline{print}-Funktionsaufrufen.

\begin{figure}[htb]
\centering
\begin{minipage}{0.6\textwidth}
\centering
\begin{lstlisting}[label={lst-calc-print-1}, caption={Rechenbeispiele}]
print(5 + 3)
print(<@\color{textcolor}f@>"Ergebnis: <@\color{keywordcolor}\{\color{black}5 - 3\color{keywordcolor}\}@>")
print(<@\color{textcolor}f@>"Erst Punkt: <@\color{keywordcolor}\{\color{black}3.7 + 5 * 3\color{keywordcolor}\}@>")
print(5 / 3)
print(2 ** 16)
\end{lstlisting}
\end{minipage}
\hfill
\begin{minipage}{0.35\textwidth}
\centering
\begin{lstlisting}[language=output, caption={Konsolenausgabe}, label={lst-calc-print-output-1}]
8
Ergebnis: 2
Punkt zuerst: 18.7
1.6666666666666667
65536
\end{lstlisting}
\end{minipage}
\end{figure}

In der Konsole (siehe \autoref{lst-calc-print-output-1}) ist zu sehen, dass Zahlen nicht immer exakt berechnet werden. Ausserdem kann man erkennen, dass Python die \say{Punkt-vor-Strich-Regel} beachtet (Rechnung in Zeile \num{3}). Formatierter Text ist auch in der Lage, eine Rechnung zu beinhalten. Die Rechnung wird automatisch ausgerechnet.

\end{example}

\begin{important}
Möchte man \say{Kommazahlen} in Rechnungen verwenden, so muss man einen Punkt für das Komma verwenden!
\end{important}

\subsection{Ganzzahlige Division, Klammern und Variablen}

Die Division mit dem doppelten Schrägstrich (eng. double forward slash) bewirkt, dass nur der \textbf{ganzzahlige Anteil} der Division als Ergebnis herauskommt. Es wird also die gewöhnliche Division berechnet und dann zur \textbf{nächsten ganzen Zahl abgerundet} (mathematische Abrundungsfunktion) benutzt. Möchte man die Reihenfolge der Berechnungen beeinflussen, dann kann man wie in der Mathematik \textbf{runde} Klammern verwenden. Natürlich kann man bei Rechnungen auch Variablen und Werte kombinieren.

\begin{figure}[htb]
\centering
\begin{minipage}{0.5\textwidth}
\centering
\begin{lstlisting}
import random as r

print(7 // 2)
print(1 // 2)
print(9.0 // 3)
a = r.randrange(1, 11)
b = r.randrange(1, 11)
ergebnis = (a + b) * (a - b)
\end{lstlisting}
\end{minipage}
\hfill
\begin{minipage}{0.45\textwidth}
\centering
\begin{lstlisting}
import random as r

a = r.randrange(1, 101)
print(<@\color{textcolor}f@>"Seitenlänge: <@\color{keywordcolor}\{\color{black}a\color{keywordcolor}\}@>")
umfang = a + a + a + a
flaeche = a ** 2
print(<@\color{textcolor}f@>"Umfang: <@\color{keywordcolor}\{\color{black}umfang\color{keywordcolor}\}@>")
print(<@\color{textcolor}f@>"Fläche: <@\color{keywordcolor}\{\color{black}flaeche\color{keywordcolor}\}@>")\end{lstlisting}
\end{minipage}
\caption{Weitere Beispiele wie man in Python rechnen kann. Bei einer Zuweisung wird immer \textbf{zuerst} die \textbf{rechte Seite} des Zuweisungsoperators ausgewertet. Das Ergebnis wird dann in der Variablen auf der \textbf{linken Seite} des Zuweisungsoperators gespeichert.}
\label{lst-calc-print-2}
\end{figure}

\begin{important}
	Im Gegensatz zur Mathematik müssen immer alle \textbf{arithmetischen Operatoren} notiert werden. Ein fehlender arithmetischer Operator, wie in folgendem Code,
	
\begin{lstlisting}
import random as r
		
x = r.randrange(1, 11)
ergebnis = 4x
\end{lstlisting}

erzeugt einen Fehler.

\end{important}

\subsection{Clean Code}

Eine weitere Regel vereinheitlicht die Darstellung einer Rechnung.

\begin{cleancode}[Leerzeichen 4]
Links \textbf{und} rechts eines \textbf{arithmetischen Operators} fügen wir je ein \textbf{Leerzeichen} ein.
\end{cleancode}

\begin{example}
Wir notieren die Addition von \num{5} und \num{3} somit wie folgt: \lstinline{5 + 3}. Falsch, im Sinne von Clean Code, wäre \lstinline{5+3}.
\end{example}

\newpage

\subsection{Arithmetischer Ausdruck}

Mathematische Rechnungen sind Programmierbefehle, die in Python eine eigene Kategorie darstellen.

\begin{definition}[Arithmetischer Ausdruck]
	Rechnungen mit \textbf{arithmetischen Operatoren} werden arithmetische Ausdrücke (eng. arithmetic expressions) genannt. Bei der Programmausführung werden arithmetische Ausdrücke stets von Python direkt \textbf{ausgewertet}. Dies bedeutet, Python ermittelt für den arithmetischen Ausdruck einen \textbf{Wert} (\say{das Ergebnis der Rechnung}). Bei arithmetischen Ausdrücken ist der Wert immer eine Zahl (entweder eine ganze Zahl oder eine Fliesskommazahl). 
\end{definition}

\begin{example}
Schauen wir uns \lstinline{(163 * 3) - (77  * 4)} im Detail an:
$$
\begin{array}[t]{c}
\begin{array}[t]{cccccccccccc} 
\underbrace{\textrm{\texttt{(}}}_{\textrm{Klammer}} & \underbrace{\textrm{\texttt{163}}}_{\textrm{Wert}} & \underbrace{\textrm{\texttt{*}}}_{\textrm{arith. Op.}} & \underbrace{\textrm{\texttt{3}}}_{\textrm{Wert}} & \underbrace{\textrm{\texttt{)}}}_{\textrm{Klammer}} & \underbrace{\textrm{\texttt{-}}}_{\textrm{arith. Op.}} & \underbrace{\textrm{\texttt{(}}}_{\textrm{Klammer}} & \underbrace{\textrm{\texttt{77}}}_{\textrm{Wert}} & \underbrace{\textrm{\texttt{*}}}_{\textrm{arith. Op.}} & \underbrace{\textrm{\texttt{4}}}_{\textrm{Wert}} & \underbrace{\textrm{\texttt{)}}}_{\textrm{Klammer}}
\end{array} \\
\underbrace{\hspace{14cm}}_{\textrm{arithmetischer Ausdruck}}
\end{array}
$$
Mit arith. Op. ist der Begriff arithmetischer Operator gemeint. Würden wir den arithmetischen Ausdruck mit einem \lstinline{print}-Funktionsaufruf kombinieren, dann würde Python in der Konsole die Zahl \num{181} ausgeben.

\begin{lstlisting}
print((163 * 3) - (77  * 4))
\end{lstlisting}

\end{example}

\section{Aufgaben}

In den folgenden Aufgaben setzen Sie sich mit den arithmetischen Operatoren auseinander.

\subsection{Aufgabe 1}

Notieren Sie ein Programm, welches das \textbf{Volumen} und den \textbf{Oberflächeninhalt} eines \textbf{Würfels} berechnet. Verwenden Sie für die Kantenlänge $a$ eine zufällige Zahl zwischen \num{1} und \num{100}. Das Volumen berechnet sich durch $a^3$ und der Oberflächeninhalt mit $6 \cdot a^2$. Verwenden Sie Variablen und eine saubere Konsolenausgabe (siehe \autoref{lst-calc-print-2}).

\fillwithgrid{\stretch{1}}

\newpage

\subsection{Aufgabe 2}

Notieren Sie ein Programm, welches ein \textbf{regelmässiges Siebeneck} zeichnet. Die Seitenlänge \num{150} soll in einer Variablen gespeichert sein. Der Drehwinkel der Turtle soll durch einen arithmetischen Operator berechnet und in einer Variablen gespeichert werden. Verwenden Sie dann eine \lstinline{for}-Schleife mit einem \lstinline{range}-Funktionsaufruf, um das \num{7}-Eck zu zeichnen.

\fillwithgrid{1.9in}

\subsection{Aufgabe 3}

Notieren Sie ein Programm, welches von drei zufälligen Zahlen das arithmetische Mittel (\say{Durchschnitt}) berechnet und in der Konsole ausgibt. Sie müssen dazu die drei Zahlen addieren und durch \num{3} teilen. Die Ausgabe in der Konsole sollte dann wie folgt lauten:

\begin{lstlisting}[language=output]
Das arithmetische Mittel von 34, 31 und 52 ist 39.0.
\end{lstlisting}

Die Zahlen sind nur ein Beispiel. Das Programm muss für drei beliebige Zahlen funktionieren.

\fillwithgrid{1.9in}

\section{Der Modulooperator}

Beim Umrechnen einer Dezimalzahl zu einer Dualzahl gibt es ein Verfahren, welches wiederholt die ganzzahlige Division mit Rest durchführt. Bei diesem Verfahren ist man insbesondere am Rest der Division interessiert, da man dadurch die Dualzahl erhält (siehe Skript über \say{Digitalisierung}). Wir können das Verfahren auch programmieren. Mit dem Modulooperator können wir unkompliziert den Rest einer Division bestimmen. \autoref{lst-modulo-example} zeigt vier Beispiele. Die dazugehörige Ausgabe ist in \autoref{lst-modulo-example-output} dargestellt.

\begin{figure}[htb]
\begin{minipage}{0.45\textwidth}
\begin{lstlisting}[label={lst-modulo-example}, caption={Das Prozentzeichen ist der Modoluooperator.}]
print(5 % 2)
print(19 % 2)
print(18 % 6)
print(20 % 9)
\end{lstlisting}
\end{minipage}
\hfill
\begin{minipage}{0.45\textwidth}
\begin{lstlisting}[language=output, label={lst-modulo-example-output}, caption={Reste der ganzzahligen Divisionen.}]
1
1
0
2
\end{lstlisting}
\end{minipage}
\end{figure}

Modulo (oder kurz mod) ist der Fachbegriff für den Rest bei einer ganzzahligen Division.

\begin{definition}[Modulooperator]
Das Prozentzeichen \lstinline{%}
ist der Modulooperator. Er ist ein arithmetischer Operator, der den \textbf{Rest} einer ganzzahligen Division berechnet.
\end{definition}

\begin{example}
Der Code \lstinline{rest = 24 % 5} 
benutzt den Modulooperator. Die Variable \lstinline{rest} speichert nach der Ausführung den Wert \num{4}. Wir können uns dies wie folgt herleiten:
\begin{enumerate}
\item Ganzzahlige Division berechnen, das heisst $24 : 5 = 4 \text{ Rest } 4$
\item Nur den Rest behalten, das heisst \num{4}.
\end{enumerate}
\end{example}

\section{Aufgaben}

In den folgenden Aufgaben setzen Sie sich mit dem Modulooperator auseinander.

\subsection{Aufgabe 1}

Notieren Sie die Konsolenausgabe, wenn man das Programm aus \autoref{lst-modulo-aufgabe-1} ausführen würde.

\begin{figure}[htb]
\begin{minipage}{0.45\textwidth}
\fillwithgrid{1.5in}
\end{minipage}
\hfill
\begin{minipage}{0.45\textwidth}
\begin{lstlisting}[label={lst-modulo-aufgabe-1}, caption={Modulooperationen.}]
print(8 % 2)
print(7 % 3)
rest = 65535 % 5
print(rest)
print(0 % 16)
print(42 % 0)
\end{lstlisting}
\end{minipage}
\end{figure}

\subsection{Aufgabe 2}

Notieren Sie ein Programm, welches eine zufällige Zahl zwischen \num{1} und \num{1000} erstellt. Für diese Zahl soll in der Konsole ersichtlich sein, ob die Zahl durch \num{3}, \num{6} oder \num{9} teilbar ist. Verwenden Sie eine \lstinline{for}-Schleife, welche die Teilbarkeit für die drei Zahlen in Kombination mit dem Modulooperator prüft. \autoref{lst-modulo-aufgabe-2} zeigt eine Beispielausgabe.

\begin{figure}[htb]
\begin{minipage}{0.55\textwidth}
\fillwithgrid{3in}
\end{minipage}
\hfill
\begin{minipage}{0.4\textwidth}
\begin{lstlisting}[language=output, label={lst-modulo-aufgabe-2}, caption={Beispielausgabe.}]
309 % 3 ergibt 0
309 % 6 ergibt 3
309 % 9 ergibt 3
\end{lstlisting}
\begin{important}[Teilbarkeit]
Eine Zahl ist durch eine andere Zahl teilbar, wenn es bei der Division keinen Rest gibt (\say{Rest ist \num{0}}).
\end{important}
\end{minipage}
\end{figure}

\newpage

\subsection{Aufgabe 3}

Das Programm aus \autoref{lst-modulo-aufgabe-3} ist gegeben. Das Ziel dieser Aufgabe lautet: welche Berechnung führt das Programm durch? Gehen Sie zur Beantwortung der Frage schrittweise vor:

\begin{enumerate}
\item Lesen Sie das Programm durch und klären Sie unklare Code-Zeilen.
\item Führen Sie das Programm aus (in die Python-Datei \graybgtexttt{unbekannte\_berechnung.py} abtippen) und beobachten Sie die Ausgabe.
\item Analysieren Sie den Schleifenkörper. Notieren Sie für \textbf{jede Code-Zeile im Schleifenkörper} den Sinn der Berechnung. Sie können direkt in das Listing schreiben.
\item Notieren Sie in \textbf{einem} Satz, was das Programm berechnet.
\item Schreiben Sie das Programm um. Sie sollen jede Variable (ausser \lstinline{zahl}) durch eine Variable mit einem \textbf{sinnvollen} Namen ersetzen. Passen Sie auch die Konsolenausgabe an. Notieren Sie das korrigierte Programm direkt hier.
\end{enumerate}

\begin{hinweis}
Es wurden absichtlich \say{schlechte} Variablennamen benutzt. An dieser Aufgabe können Sie sehen, wie schwierig es ist, den Sinn eines Programmes zu verstehen, wenn man schlechte Variablennamen benutzt.
\end{hinweis}

\begin{lstlisting}[caption={\graybgtexttt{unbekannte\_berechnung.py}}, label={lst-modulo-aufgabe-3}]
import random as r

zahl = r.randrange(1, 1000)

q = 0
r = zahl
for s in [100, 10, 1]:
    z = r // s
    q = q + z
    r = r % s
print(<@\color{textcolor}f@>"<@\color{keywordcolor}\{\color{black}zahl\color{keywordcolor}\}@> <@\color{keywordcolor}\{\color{black}q\color{keywordcolor}\}@>")
\end{lstlisting}

\fillwithgrid{\stretch{1}}